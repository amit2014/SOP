\section*{Experiences: Better software through industry and academic techniques}
I have worked in both industry and academia, and these experiences have prepared me for graduate school at UIUC.

\subsection{Beginning research at Notre Dame}
In Summer ‘11, I gained my first exposure to research in the REU program at the University of Notre Dame, working with Dr. Aaron Striegel. 
My partner and I developed a data visualization tool for use in a sociological study, utilizing an iterative design process based on feedback from graduate students and professors. 
We practiced agile software development guided by research in data visualization and HCI, a mix of industry and academic techniques. 
We ended the summer by winning the poster competition, affirming the best software is made from both industry and academic practices.

\subsection{Game development at BSU}
The following Spring, I was selected to participate in a semester-long immersive learning project at Ball State University mentored by Dr. Paul Gestwicki. 
Twelve interdisciplinary students and I were to create an educational video game for our community partner, the Children’s Museum of Indianapolis. 
For the duration of the project, we were no longer students but employees at a game development studio working 40 hours a week. 

I was appointed the Technical Lead, and my main responsibility was communication between the developers and the graphic design team. 
We utilized agile software development techniques such as daily Scrum meetings, test-driven development, and regular code reviews. 
My exposure to these techniques prompted me to take a critical look at the process of creating software, beginning my interest in software engineering research.

Since the end of the semester, I have continued to work with Dr. Gestwicki to submit the game to academic conferences such as Meaningful Play. 
We are currently conducting empirical research to better understand the impact of educational video games. 
We will interview students playing the game, perform qualitative data analysis, and publish the results. 
These findings will help game designers develop more effective educational games, yet another example of software development guided by academic research.

\subsection{Software engineering research at UIUC}
In Summer ‘12, I was accepted to the University of Illinois at Urbana-Champaign REU program working with Dr. Danny Dig, whose academic focus is software engineering. 
My project for the summer was to develop two automated refactorings in Java to support lambda expressions, which will be introduced in Java 8. 
I worked with another intern and employees from Oracle to submit the refactorings as a patch to the NetBeans IDE, an open source project used by millions of developers. 

We analyzed the code in open source projects in order to evaluate the applicability, value, and effort saved by our tool. 
This evaluation could only be accomplished through analysis of empirical data. 
Our findings showed that our refactorings were highly applicable and would provide value to developers. 
With such positive results, we decided to submit our work as a research paper to the International Conference on Software Engineering, one of the top conferences in computer science. 

By the end of the summer, our tool was accepted as a patch to an open source project and our paper was submitted to a top tier software engineering conference. 
This experience convinced me I had the mettle to succeed at a graduate program like UIUC.